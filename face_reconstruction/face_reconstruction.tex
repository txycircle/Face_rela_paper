\documentclass[UTF8,12pt]{article} % 12pt 为字号大小
\usepackage{amssymb,amsfonts,amsmath,amsthm}
%\usepackage{fontspec,xltxtra,xunicode}
%\usepackage{times}

%----------
% 定义中文环境
%----------

\usepackage{xeCJK}

\setCJKmainfont[BoldFont={SimHei},ItalicFont={KaiTi}]{SimSun}
\setCJKsansfont{SimHei}
\setCJKfamilyfont{zhsong}{SimSun}
\setCJKfamilyfont{zhhei}{SimHei}
\setCJKfamilyfont{zhkai}{KaiTi}
\setCJKfamilyfont{zhfs}{FangSong}
\setCJKfamilyfont{zhli}{LiSu}
\setCJKfamilyfont{zhyou}{YouYuan}

\newcommand*{\songti}{\CJKfamily{zhsong}} % 宋体
\newcommand*{\heiti}{\CJKfamily{zhhei}}   % 黑体
\newcommand*{\kaiti}{\CJKfamily{zhkai}}  % 楷体
\newcommand*{\fangsong}{\CJKfamily{zhfs}} % 仿宋
\newcommand*{\lishu}{\CJKfamily{zhli}}    % 隶书
\newcommand*{\yuanti}{\CJKfamily{zhyou}} % 圆体

%----------
% 版面设置
%----------
%首段缩进
\usepackage{indentfirst}
\setlength{\parindent}{2em}

%行距
\renewcommand{\baselinestretch}{1.4} % 1.4倍行距

%页边距
\usepackage[a4paper]{geometry}
\geometry{verbose,
  tmargin=2cm,% 上边距
  bmargin=2cm,% 下边距
  lmargin=3cm,% 左边距
  rmargin=3cm % 右边距
}


%----------
% 其他宏包
%----------
%图形相关
\usepackage[x11names]{xcolor} % must before tikz, x11names defines RoyalBlue3
\usepackage{graphicx}
\usepackage{pstricks,pst-plot,pst-eps}
\usepackage{subfig}
\def\pgfsysdriver{pgfsys-dvipdfmx.def} % put before tikz
\usepackage{tikz}

%原文照排
\usepackage{verbatim}

%网址
\usepackage{url}

%----------
% 习题与解答环境
%----------
%习题环境
\theoremstyle{definition} 
\newtheorem{exs}{习题}

%解答环境
\ifx\proof\undefined\
\newenvironment{proof}[1][\protect\proofname]{\par
\normalfont\topsep6\p@\@plus6\p@\relax
\trivlist
\itemindent\parindent
\item[\hskip\labelsep
\scshape
#1]\ignorespaces
}{%
\endtrivlist\@endpefalse
}
\fi

\renewcommand{\proofname}{\it{证明}}


%==========
% 正文部分
%==========

\begin{document}

\title{人脸重建相关论文整理}
\author{Xinyuan}
%\date{} % 若不需要自动插入日期,则去掉前面的注释;{ } 中也可以自定义日期格式
\maketitle
\section{多视角}
\subsection{Deep Facial Non-Rigid Multi-View Stereo}


作者基于多视角人脸图片,实现端到端的训练神经网络得到对应每一张输入图片的人脸三维重建模型。


1、人脸模型的描述方便:3DMM+非线性部分(网络)
	
	同一组输入图片有相同的非线性基,作者利用这部分非线性基来弥补3DMM模型表达能力不足的问题。利用当前重建MESH的结果,到输入图片上提取特征,并得到UVmap,以及position MAP,得到各组输入图片的基。
	
	
2、\textit{\textbf{借助多视角之间的几何一致性融合多视角信息}}。(在算法的正向计算中利用多视角信息)
	
	由于3dmm可以作为不同视角下人脸的对齐桥梁(理解为语义对齐),投影到各个视角下,能够实现\textbf{语义上的一致性},并将这个一致性loss作为后面的能量函数。
	
3、不同于用网络回归得到外参以及人脸模型的参数,作者参考传统MVS利用几何信息的方法,借助梯度下降对当前参数进行更行。\textit{\textbf{同时利用学习的方法,得到数据集中的先验信息}},可以调整学习率,以更少的迭代次数,达到最优点。

	基于2中的能量函数进行随机梯度下降算法,更新参数(外参,人脸模型参数),这里步长是mlp学习得到的。
	
	通过实验可以知道,在有限 的迭代次数只能,优化的效果要优于ADAM优化器或者固定学习率的随机梯度下降。

4、训练LOSS
	
	监督训练,包含三维点location的误差,法向量+边距的平滑项+landmark点。

5、量化结果:三维误差:1.11mm(BU3DFE),大尺度上的人脸对齐效果不好,但是好于单视角。(our dataset)
	
\subsection{Self-Supervised Monocular 3D Face Reconstruction by Occlusion-Aware Multi-view Geometry Consistency}
这篇文章主要运用多视角信息解决姿态较大的人脸重建问题,自监督

1、人脸模型:3DMM

	先天限制,3DMM表述能力。

2、多视角几何一致性

	作者对每一张输入图片进行建模,利用不同视角之间的相对外参进行一致性比较。比如说,A,B两个视角,分别可以预估外参和模型参数,计算AB视角之间的相对外参,以A视角的三维模型,AB的相对外参,找到B \textbf{视角对应点的特征以及深度},可以进行特征一致性的比较以及深度一致性的比较。注意这里还考虑到了不同视角尺度的变化,故而有一定的缩放。
	
	但是这里将几何一致性的约束放在了无监督上面,所以在算法的正向计算中,并没有引入这一点。
	
3、极线约束

	作者创新型引入\textbf{极线约束},由于landmark的对应点已知,故而可以利用landmark的对应关系,建立极线约束关系,使得对应点在极线附近,这样可以有助于大尺度外参的估计。(后来的实验中发现极线约束与三维重建的重投影是一个物理意义,都是为了重建的两条射线共面。)
	
4、共同可见部分

	在多视角中由三维模型投影到不同视角下需要计算不同视角的共同可见点。作者提出了实现方式:
	\textbf{对于target视角中,三维P点可见,则包含P点的所有三角面片,在source视角中同样可见。}
	
5、实验结果:重建细节不明显,大尺度人脸对齐效果优于单视角,三维误差:1.55mm(BU3DFE)
\end{document}